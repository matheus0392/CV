\RequirePackage{ifthen}
\newif\ifen
\newif\iffr
\newif\ifpt
\newcommand{\en}[1]{\ifen#1\fi}
\newcommand{\fr}[1]{\iffr#1\fi}
\newcommand{\pt}[1]{\ifpt#1\fi}
\newcommand{\translate}[3]{\ifen#1\fi\iffr#2\fi\ifpt#3\fi}

%updated month
\newcommand{\atualizado}[0]{\outubro 2025}
%-----------------------------
%chose one language, not two or three
%
%\entrue %English
\frtrue	%Français
%\pttrue %Português
%-----------------------------
\documentclass[12pt, a3paper, sans, final]{moderncv}
\moderncvstyle{banking} % style options are 'casual' (default), 'classic', 'oldstyle' 'fancy' and 'banking', 'contemporary', 'empty'
\moderncvcolor{blue} % color options 'blue' (default), 'orange', 'green', 'red', 'purple', 'grey', 'black' and 'burgundy'
%\definecolor{color1}{rgb}{0.0 , 0.55 , 0.55} % black,white,red,darkgrey,lightgrey,orange,burgundy, purple,lightblue,green,cerulean,darkcyan,verdigris,lightskyblue, darkred,palegreen,lightorange, lavender,firebrick,lightblack
\definecolor{color1}{rgb}{0.05, 0.50, 0.05}
%\usepackage{moderncvcolors}

\colorlet{lastnamecolor}{color1}
\colorlet{firstnamecolor}{color1}%{lastnamecolor!50}
\colorlet{namecolor}{color1}

\colorlet{headrulecolor}{color1}
\colorlet{quotecolor}{color1}
\colorlet{sectioncolor}{color1}
%\colorlet{pictureframecolor}{color1}
%\colorlet{titlecolor}{color2}
%\colorlet{addresscolor}{color2}
%\colorlet{bodyrulecolor}{color1}
%\colorlet{subsectioncolor}{color1}
%\colorlet{hintstylecolor}{color0}
% Letter
%\colorlet{letterclosingcolor}{color2}

\usepackage[utf8]{inputenc}
\usepackage[scale=0.90]{geometry}
\usepackage{fancyhdr}
\usepackage{enumitem}
\usepackage{lastpage}
\usepackage[strict]{changepage}
\usepackage{changepage}
\usepackage{framed}
%\usepackage{fontawesome}
\recomputelengths


%-----------------------------
% personal data
\newcommand{\emailum}{mnascimento0392@gmail.com}
\newcommand{\emaildois}{weddynny@hotmail.com}
\newcommand{\username}{matheus0392}
\newcommand{\nome}{Matheus}
\newcommand{\sobrenome}{da Silva Nascimento}

\address{2600, Rue de la Vendée -- \translate{Québec City}{Ville de Québec{Québec}} ,, QC}{\countryCN}
\email{\emailum}
\social[linkedin]{\username}
\social[GitHub]{\username}
\name{\nome}{\sobrenome}
%\phone[mobile]{+55 61998382002}
%\social[skype]{weddynny}
%skype:live:weddynny
%\social[gitlab]{jdoe} % not working
%\social[skype]{\emaildois} %not working
%\photo[64pt][0.4pt]{picture}
%\quote{Some quote}


% ------------- variaveis % geral
\newcommand{\textVersion}{\translate{english version.}{version française.}{versão portuguesa.}}
\newcommand{\brasil}{\translate{Brazil}{Brésil}{Brasil}}
\newcommand{\canada}{\translate{Canada}{Canada}{Canadá}}
\newcommand{\countryBR}{\brasil}
\newcommand{\countryCN}{\canada}
\newcommand{\cityBSB}{\translate{Brasilia}{Brasilia}{Brasília}}
\newcommand{\citySP}{\translate{São Paulo}{São Paulo}{São Paulo}}
\newcommand{\cityQC}{\translate{Québec}{Québec}{Québec}}
\newcommand{\janeiro}{\translate{January }{Janvier }{Janeiro }}
\newcommand{\fevereiro}{\translate{February }{Février }{Fevereiro }}
\newcommand{\marco}{\translate{March }{Mars }{Março }}
\newcommand{\abril}{\translate{April }{Avril }{Abril }}
\newcommand{\maio}{\translate{May }{Mai }{Maio }}
\newcommand{\junho}{\translate{June }{Juin }{Junho }}
\newcommand{\julho}{\translate{July }{Juillet }{Julho }}
\newcommand{\agosto}{\translate{August }{Août }{Agosto }}
\newcommand{\setembro}{\translate{September }{Septembre }{Setembro }}
\newcommand{\outubro}{\translate{October }{Octobre }{Outubro }}
\newcommand{\novembro}{\translate{November }{Novembre }{Novembro }}
\newcommand{\dezembro}{\translate{December }{Décembre }{Dezembro }}

\newcommand{\atual}{\translate{Present}{à ce jour}{Atual}}
\newcommand{\ambiente}[1]{
\begin{ident}
    \item[\textbf{\translate{Technical skills}{Compétences techniques}{Competências Técnicas}}]#1
\end{ident}
}
\newcommand{\remoto}{ (\translate{Remote}{À distance}{Remoto})}
\newcommand{\projeto}{\translate{Project}{Projet}{Projeto}}
\newcommand{\mandato}{\translate{Mandate}{Mandate}{Mandato}}

\newcommand{\desenvolvedor}{\translate{Developer}{Développeur}{Desenvolvedor}}
\newcommand{\programador}{\translate{Programmer}{Programmeur}{Programador}}
\newcommand{\analista}{\translate{Analyst}{Analyste}{Analista}}
\newcommand{\msss}{\translate{Ministry of Health and Social Services}{Ministère de la Santé et des Services sociaux}{}}



%hrefs
\newcommand{\reff}[1]{\href{#1}{\textcolor{blue}{\faExternalLink}}}
\newcommand{\refWes}{\reff{https://www.credly.com/badges/8e50fab0-e3f3-4a1f-9b0e-2e855497f250}}
\newcommand{\refEsc}{\reff{http://www.cmb.eb.mil.br/}}
\newcommand{\refFor}{\reff{https://github.com/forpdi/plataforma-for}}
\newcommand{\refUniv}{\reff{http://international.unb.br/}}
\newcommand{\refUm}{\reff{http://www.topocart.com.br/}}
\newcommand{\refDois}{\reff{http://next.unb.br/}}
\newcommand{\refTres}{\reff{https://gauge-next-challenge.vercel.app/}}
\newcommand{\refQuatro}{\reff{https://stefanini.com/en}}
\newcommand{\refCinco}{\reff{https://paradigme3.ca/}}

%seções
\newcommand{\sectionPerfil}{\translate{Profile}{Profil}{Perfil}}
\newcommand{\sectionSummary}{\translate{Summary}{Sommaire}{Sumário}}
\newcommand{\sectionCompetencia}{\translate{Professional skills}{Compétences Informatiques}{Competências}}
\newcommand{\sectionExeperiencia}{\translate{Work Experience}{Expériences Professionnelles}{Experiência Profissional}}
\newcommand{\sectionFormacao}{\translate{Education}{Formation}{Formação}}
\newcommand{\sectionComplementar}{\translate{Trainings and Certifications}{Formations Complémentaires}{Formação Complementar}}

%seção competencias
%\newcommand{\textLinguas}{
%\translate{{Languages: }{Portuguese, French and English}}{{Langues: }{Portugais, Français et Anglais}}{{Línguas: }{Francês, Inglês e Português}};
%}
%\newcommand{\textLinguagens}{
%\translate{\cvitem{Programming languages}{Java, JavaScript, React, React Native, Groovy, HTML, CSS.}}
%{\cvitem{Langages}{Java, JavaScript, React, React Native, Groovy, HTML, CSS.}}
%{\cvitem{Linguagens de programação}{Java, JavaScript, React, React Native, Groovy, HTML, CSS.}}
%}
%\newcommand{\textFerramentas}{
%\translate{\cvitem{Developer Tools}{Eclipse, Netbeans, VScode, Git, Jenkins, Docker, CMS.}}
%{\cvitem{Outils de développement}{Eclipse, Netbeans, VScode, Git, Jenkins, Docker, CMS.}}
%{\cvitem{Ferramentas de Desenvolvimento}{Eclipse, Netbeans, VScode, Git, Jenkins, Docker, CMS.}}
%}
%\newcommand{\textBD}{
%\translate{\cvitem{Database}{PostgresSQL/PostGIS, MySQL.}}
%{\cvitem{Base de données}{PostgresSQL/PostGIS, MySQL.}}
%{\cvitem{Banco de Dados}{PostgresSQL/PostGIS, MySQL.}}
%}


% seção de formação escola
\newcommand{\univ}{\translate{University of Brasilia -- UnB}{Université de Brasilia -- UnB}{Universidade de Brasília -- UnB} \refUniv}
\newcommand{\diploma}{\translate{Bachelor's degree in computer engineering}{Diplôme de baccalauréat en génie informatique}{Diploma de bacharel em engenharia de computação} \refWes}
\newcommand{\textUniv}{\cventry{\cityBSB, \countryBR}{\univ}{\diploma}{\agosto 2011-- \agosto 2016}{}{}}

\newcommand{\escola}{\translate{Military School of Brasilia}{École Militaire de Brasilia}{Colégio Militar de Brasília} \refEsc}
\newcommand{\diplomaEsc}{\translate{Elementary and Secondary school diploma}{Diplôme d'études primaires et secondaires}{Certificado de ensino fundamental e médio}}
\newcommand{\textEscola}{\cventry{\cityBSB, \countryBR}{\escola}{\diplomaEsc}{2004--2010}{}{}}

% seção de formação complementar
\newcommand{\textComplementUm}{
\cventry{2017}{\translate{Programming course 40h}{Cours d'informatique 40h}{Curso de informática 40h}, Caelum (\brasil)\hfill}{\translate{Java for Web Development}{Java pour le développement Web}{Curso Java para o desenvolvimento Web}}{}{(Spring, Git, JUnit, Maven)}{\hfill}
}

\newcommand{\textComplementDois}{
\cventry{2017}{\translate{Programming course 20h}{Cours d'informatique 20h}{Curso de informática 20h}, Caelum (\brasil)\hfill}{\translate{Develop in practice with Spring and testing}{Développer en pratique avec Spring et tests}{Curso Desenvolvendo na prática com Spring e testes}}{}{(Spring, Git, JUnit, Maven)}{\hfill}
}

\newcommand{\textComplementTres}{
\cventry{2020}{\translate{Certification}{Certification}{Certificado}}{OCA Java SE 8 Programmer (1Z0-808) \reff{https://www.credly.com/badges/2406402e-3045-4ff3-ae26-8c965c4878fa}}{}{Oracle}{ \hfill}
}

% recipient data
\newcommand{\cover}{
\recipient{Company Recruitment team}{Company, Inc.\\123 somestreet\\some city}
\date{\setembro 18, 2025}
\opening{Dear Sir or Madam,}
\closing{Yours faithfully,}
\enclosure[Attached]{curriculum vit\ae{}} % use an optional argument to use a string other than "Enclosure", or redefine \translateclname
\makelettertitle

Lorem ipsum dolor sit amet, consectetur adipiscing elit. Duis ullamcorper neque sit amet lectus facilisis sed luctus nisl iaculis. Vivamus at neque arcu, sed tempor quam.
Curabitur pharetra tincidunt tincidunt. Morbi volutpat feugiat mauris, quis tempor neque vehicula volutpat. Duis tristique justo vel massa fermentum accumsan. Mauris ante
elit, feugiat vestibulum tempor eget, eleifend ac ipsum. Donec scelerisque lobortis ipsum eu vestibulum. Pellentesque vel massa at felis accumsan rhoncus.

Suspendisse commodo, massa eu congue tincidunt, elit mauris pellentesque orci, cursus tempor odio nisl euismod augue. Aliquam adipiscing nibh ut odio sodales et pulvinar
tortor laoreet. Mauris a accumsan ligula. Class aptent taciti sociosqu ad litora torquent per conubia nostra, per inceptos himenaeos. Suspendisse vulputate sem vehicula
ipsum varius nec tempus dui dapibus. Phasellus et est urna, ut auctor erat. Sed tincidunt odio id odio aliquam mattis. Donec sapien nulla, feugiat eget adipiscing sit amet,
lacinia ut dolor. Phasellus tincidunt, leo a fringilla consectetur, felis diam aliquam urna, vitae aliquet lectus orci nec velit. Vivamus dapibus varius blandit.

Duis sit amet magna ante, at sodales diam. Aenean consectetur porta risus et sagittis. Ut interdum, enim varius pellentesque tincidunt, magna libero sodales tortor, ut
fermentum nunc metus a ante. Vivamus odio leo, tincidunt eu luctus ut, sollicitudin sit amet metus. Nunc sed orci lectus. Ut sodales magna sed velit volutpat sit amet pulvinar diam venenatis.
\makeletterclosing
\clearpage
}

\newcommand{\labelident}[1]{\textsc{#1:}}
\newlength\normalparindent
\newenvironment{ident}{\begin{list}{}
    {\renewcommand{\makelabel}{\labelident}
      \setlength{\itemindent}{0cm}
      \setlength{\leftmargin}{0.8cm}
      \setlength{\labelwidth}{0\normalparindent}
      \addtolength{\topsep}{-1\parskip}
      \setlength{\parsep}{\parskip}}}
  {\end{list}}

  \newcommand{\bul}{\color{green}\textbullet}

%\newcommand{\sectiontitle}{}
%\newcommand{\newsection}[1]{\section{#1}\renewcommand{\sectiontitle}{#1}}

%\pagestyle{fancy}
%\fancyhf{}
%\fancyhead[L]{\section{ texto (suite)}}



%\renewcommand{\footrulewidth}{0.8pt}% Footer rule
%\fancyfoot[L]{\scriptsize{Curriculum Vitae - Matheus da Silva Nascimento - {Atualizado em}{Mise à jour}\translate{Last Updated in} \abril 2020}}% Left footer
%\fancyfoot[R]{\thepage{} of \pageref{LastPage}}% Right footer


 % \fancyfoot[L]{\scriptsize{Curriculum Vitae - Matheus da Silva Nascimento - {Atualizado em}{Mise à jour}\translate{Last Updated in} \marco 2020}}% Left footer
  %\fancyfoot[R]{\thepage{} of \pageref{LastPage}}% Right footer
\pagestyle{fancy}
\fancypagestyle{capa}{
  \fancyhf{}
  \fancyfoot[L]{\scriptsize{Curriculum Vitae -- \nome \sobrenome \hspace{0cm} -- \translate{Last Updated in}{Mise à jour en}{Atualizado em} \atualizado}}% Left footer
  \fancyfoot[R]{\thepage{} \translate{of}{de}{de} \pageref{LastPage}}% Right footer
}

\fancypagestyle{resto}{
  \fancyhf{}
  \renewcommand{\headrulewidth}{0pt}
  \fancyhead[C]{\footnotesize{TOC \thepage}}
  \fancyfoot[C]{\footnotesize{TOC \thepage}}
}


% ------------- documento ------------- 
\begin{document}
%\include{paginainicial}
\pagestyle{capa}
%-----       letter       ---------------------------------------------------------
%\cover
%-----       resume       ---------------------------------------------------------
\makecvtitle
%\textVersion
% ------------- profil ------------- 
\setlength{\parskip}{\baselineskip}%
\setlength{\parindent}{0pt}%
\section{\sectionSummary}

\ \ \ \ \ \ \ \ \translate{With 8 years of experience developing software, more particularly in Java services (6 years), web platforms built with TypeScript (React and Angular -- 5 years), I also have solid experience in SQL (8 years). Passionate about technology and its possibilities, I have worked in both small and large companies, which makes me an adaptable professional, able to collaborate effectively in a team. Always ready to learn new tools and technologies, share knowledge and take on new challenges. Guided by responsibility and autonomy, I'm detail-oriented and eager to solve problems.}{Avec 8 ans d'expérience dans le développement de logiciels, plus particulièrement en services Java (6 ans), plateformes web construites avec TypeScript (React et Angular – 5 ans), j'ai également une solide expérience en SQL (8 ans). Passionné par la technologie et ses possibilités, j'ai travaillé dans des petites et grandes entreprises, ce qui fait de moi un professionnel adaptable, capable de collaborer efficacement en équipe. Toujours prêt à apprendre de nouveaux outils et technologies, partager des connaissances et relever de nouveaux défis. Guidé par la responsabilité et l'autonomie, je suis attentif aux détails et désireux de résoudre des problèmes.}{}
%\item[\bul] {{2 ans d'expérience en développement WEB et 1 an en programmation de scripts Groovy et SQL}\translate{2 years of experience in WEB development and 1 year in programming Groovy and SQL scripts}{2 anos de experiência em desenvolvimento WEB e 1 ano em programação de scripts Groovy e SQL}};
%\item[\bul] {{Facilité d'apprentissage}\translate{Ease to learning}{Facilidade de aprendisagem}};
%\item[\bul] {{Capacité à travailler en équipe}\translate{Ability to work in a team}{Capacidade de trabalho em equipe}};
%\item[\bul] {{Ouvert à nouveaux défis}\translate{Open to new challenges}{Aberto à novos desafios}};
%\item[\bul] {\textLinguas}
%\end{itemize}
% ------------- competencias ------------- 
%\section{\sectionCompetencia}
%\textLinguagens
%\textFerramentas
%\textBD
% ------------- experiencia ------------- 
\section{\sectionExeperiencia}

\cventry{\cityQC, \countryCN \remoto}{Paradigme3 Inc. \refCinco}{\translate{\programador \hspace{0cm} \analista}{\analista \hspace{0cm} \programador}{\analista \hspace{0cm} \programador}}{\outubro 2022 -- \atual}{\pt{}\fr{Services-conseils en TI}\en{IT Consulting Services}}{}{
    \begin{adjustwidth}{1cm}{0cm}
        \underline{\projeto: Modernization WSTAT -- Santé Québec (\junho 2024 -- \outubro 2025)} -- \textit{\translate{Clinical Statistics for Windows® }{Statistiques cliniques pour Windows®}{}}. \translate{}{Dans ce projet j’ai:}{}   %-- \translate{ WSTAT is a tool for entering and compiling service delivery hours (HPS). WStat was developed and commissioned in 1994 and this project aims to
%renew with more modern technologies in a cloud computing environment}{est un outil de saisie et de compilation des heures de prestation de services (HPS). WStat a été développé et mis en service en 1994 et ce projet a le but de le renouveler avec des technologies plus modernes dans un environnement infonuagique}{}.
        \medskip
        \begin{itemize}[leftmargin=1cm, parsep=0cm, topsep=-0.25cm]
            \item[\bul]{\translate{Work in collaboration with functional analysts, following specifications, for the creation of a modern web platform based in cloud services}{Travailler en collaboration avec les analystes fonctionnels, en suivant les spécifications, pour la création des pages web avec Angular}{Trabalhar em sintonia com os analistas funcionais, seguindo as especificações, para a criação das páginas web com Angular};}
            \item[\bul]{\translate{Deploy the application on different environments in Azure}{Effectuer le déploiement de l'application sur différents environnements sur Azure}{Fazer o deploy do aplicativo em diferentes ambientes no Azure};}
             \item[\bul]{\translate{Create a modular backend structure to decouple unrelated dependencies and improve development and deployments}{Créer une structure backend modulaire pour découpler les dépendances non liées et améliorer le développement et les déploiements}{}}.
        \end{itemize}
        \bigskip
        \ambiente{Java 17 (Quarkus, Maven, JUnit, Hibernate), Angular (TypeScript, PrimeNg, Bootstrap),  Microsoft SQL Server, Git, Pull Request review, Docker, Kubernetes, Swagger, Azure DevOps, REST, Agile, Azure Data Studio, VScode, IntelliJ, Microsoft Teams.}
        \bigskip
        \underline{\projeto: SIPAD -- MSSS (\outubro 2022 -- \junho 2024)} -- \textit{\translate{Information system for people with a
deficiency}{Système d'information pour les personnes ayant une déficience}{}}. \translate{}{Dans ce projet j’ai:}{}   %-- \translate{}{}{}.
        \medskip
        \begin{itemize}[leftmargin=1cm, parsep=0cm, topsep=-0.25cm]
            \item[\bul]{\translate{Participated in the maintenance and bug fixing of the platform with the Java language, framework Struts, SQL scripts and JasperReports templates}{Participé à la maintenance et à la correction des bugs de la plateforme avec le langage Java, framework Struts, scripts SQL et modèles JasperReports}{Participou na manutenção e correção de bugs da plataforma com a linguagem Java, framework Struts, scripts SQL e templates JasperReports};}
            \item[\bul]{\translate{Designed tests with JMeter and evaluations with VisualVM to find low performance and crash points}{Conçu des tests avec JMeter et des évaluations avec VisualVM pour trouver les points de faible performance et de blocage}{Projetou testes com JMeter  e avaliações com  VisualVM para achar pontos de baixa performance e de travamento}}.
            % \item[\bul]{\translate{}{}{}}
        \end{itemize}
        \ambiente{Java (Maven, Struts 1.3, Tomcat), Oracle SQL Developer, JMeter, iReports, VisualVM, Git, Eclipse, GitLab, Sourcetree, Microsoft Teams, Agile, Jira.} %,
    \end{adjustwidth}
}
\bigskip
\bigskip
% ------------- experiencia ------------- 
\cventry{\cityBSB, \countryBR \remoto}{Stefanini Group \refQuatro}{FullStack \desenvolvedor}{\junho 2021 -- \outubro 2022}{\translate{Global Tech Consultancy}{Conseil en stratégie digitale}{}}{}{
    \begin{adjustwidth}{1cm}{0cm}
        \underline{\projeto: \translate{Bank of}{Banque du}{Banco do} \brasil} -- \translate{One of the largest banks in Brazil with more than 78 million customers. Internal platform for measurements and results}{Une des plus grandes banques au Brésil comptant plus de 78 millions de clients%Plateforme interne pour les mesures et résultats
        }{}. \translate{}{Dans ce projet j’ai:}{} 
        \medskip
        \begin{itemize}[leftmargin=1cm, parsep=0.1cm, topsep=-0.25cm]
            \item[\bul]{\translate{Developed microservices and organized the operations and their dependencies in a catalog}{Développé microservices et organisé les opérations et leurs dépendances dans un catalogue}{Desenvolveu microsserviços e organizou as operações e suas dependências em um catálogo};}
            \item[\bul]{\translate{Developed new features and migrated others from AngularJS to Angular}{Développé de nouvelles fonctionnalités et migré d'autres d'AngularJS à Angular}{Desenvolveu novas features e migrou outras de AngularJS para Angular};}
            \item[\bul]{\translate{Ensured code quality with SonarQube analysis and tests using Karma/JUnit}{Assuré de la qualité du code avec les analyses de SonarQube et test avec Karma/JUnit}{Garantiu qualidade de código com análises do SonarQube e teste com Karma/JUnit};}
            \item[\bul]{\translate{Managed product versions and deployments}{Géré des versions et des déploiements du produit}{Gerenciou versões e implantações do produto}}.
        \end{itemize}
        \ambiente{Java (Quarkus, Maven, JUnit), Angular (TypeScript, PrimeNg, PrimeFlex, Karma, JasmineJS), AngularJS (JavaScript, Bootstrap, Grunt, Karma, JasmineJS),  Docker,  Kubernetes, Rancher, ArgoCD, SonarQube, Artifactory, Highcharts, Grafana, Git, Swagger, Jenkins, VScode, Swagger, Eclipse, Git, Gitlab, DBeaver, Agile, VMware Virtual Desktop Infrastructure (VDI), Microsoft Teams.}
    \end{adjustwidth}
}
\bigskip
\bigskip

% ------------- experiencia ------------- 
\cventry{\citySP, \countryBR \remoto}{Gauge -- Stefanini Group \refTres}{FullStack \desenvolvedor}{\setembro 2020 -- \junho 2021}{\translate{Digital Consulting}{Conseil en stratégie digitale}{}}{}{
    \begin{adjustwidth}{1cm}{0cm}
        \underline{\projeto: Yazigi Yconnect} -- \translate{Online platform belonging to the Brazil Pearson group for foreign language courses%The platform has about 70k students per year and receives more than 5M access per month
        }{Plateforme en ligne pour les cours de langues étrangères appartenant au groupe Pearson%La plateforme compte environ 70k étudiants par an et reçoit plus de 5M d'accès par mois
        }{Plataforma online para cursos de línguas estrangeiras que pertence ao grupo Pearson%. A plataforma possui cerca de 70k alunos por ano e recebe mais de 5M de acesso por mês
        }. \translate{}{Dans ce projet j’ai:}{}
        \begin{itemize}[leftmargin=1cm, parsep=0.1cm, topsep=0.1cm]
             \item[\bul]{\translate{Created, modified and corrected the templates for the register of new activities}{Créé, modifié et corrigé les modèles pour le registre des nouvelles activités}{};}
             \item[\bul]{\translate{Created SQL correction scripts for activities and media in the database}{Créé des scripts SQL de correction pour les activités et médias à la base de données}{Criou scripts SQL de correção para as atividades e mídias no banco de dados};}
            \item[\bul]{\translate{Participated in the construction of a quiz system for initial assessment of new students}{Participé à la construction d'un système de quiz pour l'évaluation initiale des nouveaux élèves}{Participou da construção de um sistema de quiz para avaliação inicial de novos alunos}}.
        \end{itemize}
        \ambiente{ReactJS (TypeScript, Axios, Bootstrap, HTML, CSS, styled-components), .NET, Microsoft SQL Server, Bitbucket, Gitflow, Pull Request review, Agile Asana, VScode, TaskRow, JIRA.} %
        
        \break
        %\bigskip
        \underline{\projeto: Wizard Wiz.Tab} -- \translate{Android application belonging to the Pearson Brazil  group for foreign language courses% The application has more than 100k downloads
        }{Application Android pour les cours de langues étrangères appartenant au groupe Pearson%L'application a plus de 100k téléchargements
        }{Aplicativo Android para cursos de idiomas estrangeiros pertencente ao grupo Pearson%. O aplicativo possui mais de 100k downloads
        }. \translate{}{Dans ce projet j’ai:}{}
        \medskip
        \begin{itemize}[leftmargin=1cm, parsep=0.1cm, topsep=-0.25cm]
             \item[\bul]{\translate{Answered customer's requests recorded in ServiceNow regarding the production environment and made corrections to the Wiz.Tab mobile app}{Répondu aux demandes des clients enregistrés sur ServiceNow par rapport à l'environnement de production et effectuer des corrections à l'application mobile Wiz.Tab}{Atendeu às solicitações dos clientes registradas no ServiceNow em relação ao ambiente de produção e fazer correções para o aplicativo móvel Wiz.Tab};}
            \item[\bul]{\translate{Updated the version in the Google Play Store and made deploys in AWS serverless}{Mis à jour la version dans le Google Play Store et déployé sur AWS serverless}{Atualizou a versão na Google Play Store e fez implantações no AWS serverless}}.
        \end{itemize}
        \bigskip
        \ambiente{ React Native, ServiceNow, AWS Lambda, Asana, VScode, TaskRow, JIRA.} %
    \end{adjustwidth}
}
\bigskip
\bigskip

% ------------- experiencia ------------- 
\cventry{\cityBSB, \countryBR}{NExT UnB \refDois }{\translate{\programador \hspace{0cm} \analista}{\analista \hspace{0cm} \programador}{\analista \hspace{0cm} \programador}}{\agosto 2018 -- \setembro 2020}{\translate{R\&D Center for Excellence and Organizational Transformation (UnB)}{Centre de recherche et développement pour le secteur public (UnB)}{Núcleo de Pesquisa e Desenvolvimento para Excelência e Transformação do Setor Público (UnB)}}{}{
    \begin{adjustwidth}{1cm}{0cm}
    %\cventry{1}{}{projeto evobiz ticshealth}{agosto 2018 -- janeiro 2020 }{}{6}{7}
        \underline{\projeto: Evobiz Ticshealth (\janeiro 2020 -- \setembro 2020)} -- \translate{TicsHealth is a technology from Evobiz to promote efficiency in health organizations}{TicsHealth est une technologie d'Evobiz pour promouvoir l'efficacité dans les organisations de santé}{O TicsHealth é uma tecnologia da Evobiz para promover eficiência em organizações da Saúde}. \translate{}{Dans ce projet j’ai:}{}
        \medskip
        \begin{itemize}[leftmargin=1cm, parsep=0.1cm, topsep=-0.25cm]
            \item[\bul]{\translate{Participated in the architecture planning and development of web and mobile platforms}{Participé à la planification de l'architecture et au développement des plateformes web et mobile}{Participou do planejamento da arquitetura e do desenvolvimento das plateformas Web e mobile};}
            \item[\bul]{\translate{Integrated the platform with the Philips Tasy and InterSystem Healthcare systems to expand the application for multiple customers}{Fait l'intégration avec les systèmes Philips Tasy et InterSystem Healthcare pour étendre l'application à de multiples clients}{Efetuou a integraçãoda plataforma com os sistemas Philips Tasy et InterSystem Healthcare para expandir a aplicação para múltiplos clientes};}
            \item[\bul]{\translate{Created notification mechanisms via mobile application, email, SMS, and the platform's internal messaging system}{Créé des mécanismes de notification via l'application mobile, par courriel, SMS et le système de messagerie interne de la plateforme}{Criou mecanismos de notificações via aplicativo mobile, email, SMS e sistema de mensagens interna da plataforma}};
            \item[\bul]{\translate{Reviewed Pull Requests and led other less experienced developers}{Revue les Pull Requests et guidé d'autres développeurs moins expérimentés}{Revisou Pull Requests e liderou outros desenvolvedores menos experientes}}.
        \end{itemize}
        \bigskip
        \ambiente{Java (Spring Boot, JPA, Maven, Criteria Query), ReactJS \& React Native (Axios, Bootstrap), PostgreSQL, REST, Docker, Git, Firebase, Eslint, Pull Request review, JavaScript, Eclipse, VScode, Postman.}
        \bigskip        
        \underline{\projeto: PlataformaFor (\agosto 2018 -- \janeiro 2020)} -- \translate{Open-source \refFor \hspace{0cm} tools to automate the administrative and budgetary processes of universities and public institutions% Currently 47 federal universities and 25 federal institutes use the platforms
        }{Outils open-source \refFor \hspace{0cm} pour automatiser les processus administratifs et budgétaires des universités et institutions publiques}{Ferramentas open-source \refFor \hspace{0cm} para automatizar os processos administrativos e orçamentários des universidades e instituições públicas}. \translate{}{Dans ce projet j’ai:}
        \medskip
        \begin{itemize}[leftmargin=1cm, parsep=0.1cm, topsep=-0.25cm]
            \item[\bul]{\translate{Worked on the conception of the platform and on modeling the database based on the specifications}{Travaillé sur la conception de la plateforme et sur le processus de modélisation de la base de données à partir des spécifications}{Atuei na concepção da plataforma e no processo de modelagem do banco de dados à partir das especificações};}
            \item[\bul]{\translate{Optimized slow functions, duplicated requests to the backend and database and made corrections of documented \textit{issues}}{Optimisé les fonctions lentes, les requêtes en double vers le back-end et la base de données et effectué des corrections de \textit{issues} documentés}{Otimizei funções lentas, requisições duplicadas ao backend e ao banco de dados e efetuei correções de \textit{issues} documentadas};}
            \item[\bul]{\translate{Used JMeter to detect and correct bottlenecks and slowdowns on the platform when simulating multiple concurrent users}{Utilisé JMeter pour détecter et corriger les goulots d'étranglement et les ralentissements sur la plateforme en simulant plusieurs utilisateurs simultanés}{Utilisei JMeter para econtrar e corrigir pontos de gargalo e lentidão na plataforma ao simular vários usuários simultaneos};}
            \item[\bul]{\translate{Developed tools to generate PDF reports and to generate backup files as a way of importing/exporting data on the platform.}{Développé des outils pour générer des rapports PDF et pour générer des fichiers de backup comme moyen d'importation/exportation des données sur la plateforme.}{Desenvolvi ferramentas para gerar relatórios PDF e para gerar arquivos de backup como forma de importação/exportação dos dados na plataforma}}
        \end{itemize}
        \bigskip
        \ambiente{Java (Maven, Hibernate, JMeter), ReactJS (Bootstrap, AJAX, HTML, CSS), REST, MySQL, Git, Eclipse, VScode, Postman, Slack.}
    \end{adjustwidth}
%\hspace{1cm}\ambiente{Java (Spring Boot, Spring MVC, Spring Data JPA, Maven, Criteria Query), ReactJS \& React Native (JavaScript, Axios, Bootstrap, NPM), PostgreSQL, REST, Git, Eclipse, VScode, Docker, Postman, Firebase.}\\
%\hspace{1cm}\ambiente{Java (Vraptor4, Maven, Wildfly, Hibernate, HSQL, JMeter), ReactJS(JavaScript, Bootstrap, AJAX), REST, MySQL, Git, GitHub, Eclipse, VScode, Postman, Slack, Joomla}
}
\bigskip
\bigskip

% ------------- experiencia ------------- 
\cventry{\cityBSB, \countryBR}{Topocart \refUm}{\programador}{\setembro 2017 -- \agosto 2018}{\translate{Data collection; geoprocessing; topography}{Entreprise de collecte de données; géotraitement; topographie}{Empresa de levantamento de dados, geoprocessamento, topografia}.}{}{
    \begin{adjustwidth}{1cm}{0cm}
        \underline{\projeto: Topovision} -- \translate{Geographic system to measure demographic information}{Système géographique pour mesurer les informations démographiques}{Sistema geográfico para medir informações demográficas}. \translate{}{Dans ce projet j’ai:}{}
        \medskip
        \begin{itemize}[leftmargin=1cm, parsep=0.1cm, topsep=-0.25cm]
            \item[\bul]{\translate{Created complex literal and spatial queries to combine custom filters for the platform Topovision}{Crée des requêtes littérales et spatiales complexes pour combiner des filtres personnalisés pour la plateforme Topovision}{Criou consultas literais e espaciais complexas para combinar filtros personalizados para a plataforma Topovision};}
            \item[\bul]{\translate{Created automated scripts to verify the integrity of collected data}{Créé des scripts automatisés pour vérifier l'intégrité des données collectées}{};}
             \item[\bul]{\translate{Deployed a Jenkins server to automate the analysis and increase speed in data correction by internal employees}{Implanté un serveur Jenkins pour automatiser l'analyse et augmenter la vitesse de correction des données par les employés internes}{};}
             \item[\bul]{\translate{Deployed a local Git server and migrated all the projects from subversion}{Implanté un serveur Git local et migré tous les projets de Subversion}{};}
             \item[\bul]{\translate{Planned unit and integration automated tests in the application}{Planifié des tests automatisés unitaires et d'intégration dans l'application}{}}.
        \end{itemize}
        \bigskip
    \ambiente{Groovy (Grails), JavaScript (jQuery, JasmineJS, HTML), Selenium, PostgreSQL, PostGIS, Jenkins, Subversion, Git, JasperReports, QGIS, Linux, Netbeans, PuTTY, Linux.}
    \end{adjustwidth}
}
\bigskip

% ------------- experiencia ------------- 
%\cventry{\cityBSB, \countryBR}{Superior Tribunal de Justiça - STJ {\reff{http://international.stj.jus.br/}} }{{Estágio}{Stage}\translate{Internship}}{\maio 2014 -- \dezembro 2015}{\translate{Superior Court of Justice}{Supérieur Court de Justice}{}.}{}{
%    \begin{itemize}[leftmargin=1.8cm,  parsep=0cm,  topsep=-0.25cm]
%    \item[\bul] {{Acompanhar o suporte a usuários e clientes na utilização de software}{Accompagner le support aux utilisateurs et aux clients dans l'utilisation du logiciel}\translate{Support the users and customers in the use of the software}};
%    \item[\bul] {{Acompanhamento das instalações de redes, micros e comunicação de dados}{Accompagnement d'installation de réseaux, d'ordinateurs et de la communication de données}\translate{Support for the installation of networks, computers and data communication}};
%    \item[\bul] {{Realizar manutenção de equipamentos conectados em rede}{Réaliser la maintenance d'équipements connectés à internet}\translate{Carry out maintain equipment connected to the internet}};
%    \end{itemize}}
% ------------- Formação escolar ------------- 
\section{\sectionFormacao}
\textUniv
%\textEscola
% ------------- Línguas ------------- 
%\section{Languages}
%\cvitemwithcomment{Portugais}{Natif}{}
%\cvitemwithcomment{Anglais}{ Intermediaire}{}
%\cvitemwithcomment{Français}{Debutant}{}
% ------------- Certificados e Treinamentos ------------- 
\section{\sectionComplementar}
\textComplementTres
%\textComplementDois
%\textComplementUm
% ----------------------------------------------
\end{document}


\vspace{5mm}
\vfill
\smallskip
\medskip
\bigskip
\qquad
\hfill
\hspace{5mm}
\break


linkar cada experiencia com as ferramentas?


%Concevoir
%Développer
%Effectuer
%Réaliser
%Former
%Programmer
%Générer
%Appoyer


%https://ctan.mirror.globo.tech/macros/latex/contrib/moderncv/manual/moderncv_userguide.pdf
%https://github.com/xdanaux/moderncv/blob/master/examples/template.tex
% ou http://linorg.usp.br/CTAN/fonts/fontawesome/doc/fontawesome.pdf
%https://ctan.mirror.globo.tech/macros/latex/contrib/moderncv/manual/moderncv_userguide.pdf
%https://ctan.mirror.convexic.com/macros/latex/contrib/enumitem/enumitem.pdf

%\section{Autres Informations}
%\begin{cvcolumns}
%  \cvcolumn{Category 1}{\begin{itemize}\item[\bul] Person 1\item[\bul] Person 2\item[\bul] Person 3\end{itemize}}
%  \cvcolumn{Category 2}{Amongst others:\begin{itemize}\item[\bul] Person 1, and\item[\bul] Person 2\end{itemize}(more upon request)}
%  \cvcolumn[0.5]{All the rest \& some more}{\textit{That} person, and \textbf{those} also (all available upon request).}
%\end{cvcolumns}


%exemplo de seção
%\section{Profil}
%\cvitem{category 1}{XXX, YYY, ZZZ}{category 4}{XXX, YYY, ZZZ}
%\cvdoubleitem{category 2}{XXX, YYY, ZZZ}{category 5}{XXX, YYY, ZZZ}
%\cvitemwithcomment{Portugais}{Natif}{}
%\cventry{período}{empresa}{cargo}{cidade,país}{}
%{resumo empresa.\newline{}\hyperlink{link/}{\textcolor{blue}{same link}}\newline{}%
%\begin{itemize}
%\item[\bul] Programmer 2, with sub-achievements:
% \begin{itemize}%
% \item[\bul] Effectuer consultas em postgresql/postgis;
% \end{itemize}
%\end{itemize}
%environnements: etc.
%\newline{}%}


%Inserts a blank space that will stretch accordingly to fill the vertical space available. That's why the line "Text at the bottom of the page." is moved to the bottom, and the rest of the space is filled in.There are other three commands commonly used to insert vertical blank spaces


% exemplo
\cventry{\citySP, \countryBR \remoto}{Stefanini Group \refQuatro}{FullStack \desenvolvedor}{\junho 2021 -- \outubro 2022}{\translate{Global Tech Consultancy}{Conseil en stratégie digitale}{}}{}{
    \vspace{1mm}
    \begin{adjustwidth}{1cm}{0cm}
        \medskip
        \underline{\projeto: \translate{Bank of}{Banque du}{Banco do} \brasil -}
        \begin{itemize}[leftmargin=1cm, parsep=0.1cm, topsep=-0.25cm]
            \item[\bul]{\translate{}{}{};}
            \item[\bul]{\translate{}{}{};}
        \end{itemize}
        \ambiente{AngularJS (JavaScript, Bootstrap, Grunt, Karma, JasmineJS), Angular (TypeScript, PrimeNg, PrimeFlex, Karma, JasmineJS), Java (Quarkus, Maven, JUnit), Docker, Artifactory, Kubernetes, Rancher, ArgoCD, VScode, SonarQube, Eclipse, Git, Gitlab, Swagger, Highcharts, DBeaver, REST, Agile, VMware Virtual Desktop Infrastructure (VDI), Jenkins,  Microsoft Teams. }
    \end{adjustwidth}
}
\bigskip
\bigskip