\RequirePackage{ifthen}
\newif\ifen
\newif\iffr
\newif\ifpt
\newcommand{\en}[1]{\ifen#1\fi}
\newcommand{\fr}[1]{\iffr#1\fi}
\newcommand{\pt}[1]{\ifpt#1\fi}

\documentclass[12pt,a3paper,sans, final]{moderncv}
\moderncvstyle{banking}
\moderncvcolor{burgundy}

%\definecolor{color1}{rgb}{0.0 , 0.55 , 0.55} % dark grey
%\definecolor{color0}{rgb}{0.55 , 0.05 , 0.55} % dark grey



\usepackage[utf8]{inputenc}
\usepackage[scale=0.90]{geometry}

\usepackage{fancyhdr}
\pagestyle{fancy}

\usepackage{enumitem}
%\usepackage{fontawesome}
\usepackage{lastpage}
\recomputelengths
%-----------------------------
%chose one language, not both or tre
\entrue %English
%\frtrue	%Français
%\pttrue %Português
%-----------------------------

% personal data
\newcommand{\emailum}{mnascimento0392@gmail.com}
\newcommand{\emaildois}{weddynny@hotmail.com}
\newcommand{\username}{matheus0392}
\newcommand{\nome}{Matheus}
\newcommand{\sobrenome}{da Silva Nascimento}

\en{\address{2600, Rue de la Vendée}{Québec, QC}{\countryCN}}
\fr{\address{2600, Rue de la Vendée}{Québec, QC}{\countryCN}}
\pt{\address{2600, Rue de la Vendée}{Québec, QC}{\countryCN}}
%\phone[mobile]{+55 61998382002}
\email{\emailum}
\social[linkedin]{\username}
\social[github]{\username}
\name{\nome}{\sobrenome}
%\social[skype]{weddynny}
%skype:live:weddynny
%\social[gitlab]{jdoe} % not working
%\social[skype]{\emaildois} %not working
%\photo[64pt][0.4pt]{picture}
%\quote{Some quote}


% ------------- variaveis
\newcommand{\textVersion}{\en{english version.}\fr{version française.}\pt{versão portuguesa.}}
% geral
\newcommand{\brasil}{\en{Brazil}\fr{Brésil}\pt{Brasil}}
\newcommand{\canada}{\en{Canada}\fr{Canada}\pt{Canadá}}
\newcommand{\countryBR}{\brasil}
\newcommand{\countryCN}{\canada}
\newcommand{\cityBSB}{\pt{Brasília}\en{Brasilia}\fr{Brasilia}}
\newcommand{\citySP}{\pt{São Paulo}\en{São Paulo}\fr{São Paulo}}
\newcommand{\cityQC}{\pt{Québec}\en{Québec}\fr{Québec}}
\newcommand{\janeiro}{\en{January }\fr{Janvier }\pt{Janeiro }}
\newcommand{\fevereiro}{\en{February }\fr{Février }\pt{Fevereiro }}
\newcommand{\marco}{\en{March }\fr{Mars }\pt{Março }}
\newcommand{\abril}{\en{April }\fr{Avril }\pt{Abril }}
\newcommand{\maio}{\en{May }\fr{Mai }\pt{Maio }}
\newcommand{\junho}{\en{June }\fr{Juin }\pt{Junho }}
\newcommand{\julho}{\en{July }\fr{Juillet }\pt{Julho }}
\newcommand{\agosto}{\en{August }\fr{Août }\pt{Agosto }}
\newcommand{\setembro}{\en{September }\fr{Septembre }\pt{Setembro }}
\newcommand{\outubro}{\en{October }\fr{Octobre }\pt{Outubro }}
\newcommand{\novembro}{\en{November }\fr{Novembre }\pt{Novembro }}
\newcommand{\dezembro}{\en{December }\fr{Décembre }\pt{Dezembro }}
\newcommand{\atual}{\en{Present}\fr{à ce jour}\pt{Atual}}
%\newcommand{\ambiente}{\textbf{\en{Skills}\fr{Environnements}\pt{Ambiente}}
\newcommand{\ambiente}[1]{\textbf{\en{Skills}\fr{Environnements}\pt{Ambiente}}: #1 \vspace{1.0cm}}
\newcommand{\ambientenojump}[1]{\textbf{\en{Skills}\fr{Environnements}\pt{Ambiente}}: #1}
\newcommand{\remoto}{ (\en{Remote}\fr{À distance}\pt{Remoto})}
\newcommand{\desenvolvedor}{\pt{Desenvolvedor}\fr{Développeur}\en{Developer}}
\newcommand{\programador}{\en{Programmer}\fr{Programmeur}\pt{Programador}}

%seções
\newcommand{\sectionPerfil}{\fr{Profil}\pt{Perfil}\en{Profile}}
\newcommand{\sectionCompetencia}{\fr{Compétences Informatiques}\pt{Competências} \en{Professional skills}}
\newcommand{\sectionExeperiencia}{\fr{Expériences Professionnelles}\en{Work Experience}\pt{Experiência Profissional}}
\newcommand{\sectionFormacao}{\fr{Formation}\en{Education}\pt{Formação}}
\newcommand{\sectionComplementar}{\fr{Formations Complémentaires}\en{Trainings and Certifications
}\pt{Formação Complementar}}


%seção competencias
%\newcommand{\textLinguas}{
%\en{{Languages: }{Portuguese, French and English}}\fr{{Langues: }{Portugais, Français et Anglais}}\pt{{Línguas: }{Francês, Inglês e Português}};
%}
%\newcommand{\textLinguagens}{
%\en{\cvitem{Programming languages}{Java, JavaScript, React, React Native, Groovy, HTML, CSS.}}
%\fr{\cvitem{Langages}{Java, JavaScript, React, React Native, Groovy, HTML, CSS.}}
%\pt{\cvitem{Linguagens de programação}{Java, JavaScript, React, React Native, Groovy, HTML, CSS.}}
%}
%\newcommand{\textFerramentas}{
%\en{\cvitem{Developer Tools}{Eclipse, Netbeans, VScode, Git, Jenkins, Docker, CMS.}}
%\fr{\cvitem{Outils de développement}{Eclipse, Netbeans, VScode, Git, Jenkins, Docker, CMS.}}
%\pt{\cvitem{Ferramentas de Desenvolvimento}{Eclipse, Netbeans, VScode, Git, Jenkins, Docker, CMS.}}
%}
%\newcommand{\textBD}{
%\en{\cvitem{Database}{PostgresSQL/PostGIS, MySQL.}}
%\fr{\cvitem{Base de données}{PostgresSQL/PostGIS, MySQL.}}
%\pt{\cvitem{Banco de Dados}{PostgresSQL/PostGIS, MySQL.}}
%}


% seção de formação escola
\newcommand{\univ}{\en{University of Brasilia - UnB}\fr{Université de Brasilia - UnB}\pt{Universidade de Brasília - UnB} \href{http://international.unb.br/}{\textcolor{blue}{\faExternalLink}}}
\newcommand{\diploma}{\en{Bachelor's degree in computer engineering}\fr{Diplôme de baccalauréat en génie informatique}\pt{Diploma de bacharel em engenharia de computação} \href{https://www.credly.com/badges/8e50fab0-e3f3-4a1f-9b0e-2e855497f250}{\textcolor{blue}{\faExternalLink}}}
\newcommand{\textUniv}{\cventry{}{\univ}{\diploma}{\agosto 2011-- \agosto 2016}{}{}}


%\newcommand{\hrefesc}{\href{http://www.cmb.eb.mil.br/}{\textcolor{blue}{\faExternalLink}}}
%\newcommand{\textEscola}{
%\en{\cventry{2004--2010}{Military School of Brasilia \hrefesc}{Secondary grade diploma}{\cityBSB, \countryBR}{}{\hfill}}
%\fr{\cventry{2004--2010}{École Militaire de Brasilia \hrefesc}{Diplôme d’études secondaires}{\cityBSB, \countryBR}{}{\hfill}}
%\pt{\cventry{2004--2010}{Colégio Militar de Brasília \hrefesc}{Ensino Fundamental e Médio}{\cityBSB, \countryBR}{}{\hfill}}
%}

% seção de formação complementar
\newcommand{\textComplementUm}{
\en{\cventry{2017}{Programming course 40h, Caelum (\brasil) \hfill}{Java for Web Development}{}{(Java, Spring MVC, Servlets, JDBC, Hibernate)}{\hfill}}
\fr{\cventry{2017}{Cours d'informatique 40h, Caelum (\brasil) \hfill}{Java pour le développement Web}{}{(Java, Spring MVC, Servlets, JDBC, Hibernate)}{\hfill}}
\pt{\cventry{2017}{Curso de informática 40h, Caelum (\brasil) \hfill}{Java para o desenvolvimento Web}{}{(Java, Spring MVC, Servlets, JDBC, Hibernate)}{\hfill}}
}

\newcommand{\textComplementDois}{
\en{\cventry{2017}{Programming course 20h, Caelum (\brasil)\hfill}{Develop in practice with Spring and testing}{}{(Spring, Git, JUnit, Maven)}{\hfill}}
\fr{\cventry{2017}{Cours d'informatique 20h, Caelum \brasil)\hfill}{Développer en pratique avec Spring et tests}{}{(Spring, Git, JUnit, Maven)}{\hfill}}
\pt{\cventry{2017}{Curso de informática 20h, Caelum (\brasil)\hfill}{Curso Desenvolvendo na prática com Spring e testes}{}{(Spring, Git, JUnit, Maven)}{\hfill}}
}

\newcommand{\textComplementTres}{
\cventry{2020}{\en{Certification}\fr{Certification}\pt{Certificado}}{OCA Java SE 8 Programmer (1Z0-808) {\href{https://www.credly.com/badges/2406402e-3045-4ff3-ae26-8c965c4878fa}{\textcolor{blue}{\faExternalLink}}}}{}{Oracle}{ \hfill}
}


%\newcommand{\sectiontitle}{}
%\newcommand{\newsection}[1]{\section{#1}\renewcommand{\sectiontitle}{#1}}

%\pagestyle{fancy}
%\fancyhf{}
%\fancyhead[L]{\section{ texto (suite)}}

\newcommand{\atualizado}[0]{\pt{Atualizado em}\fr{Mise à jour}\en{Last Updated in} \setembro 2025}

%\renewcommand{\footrulewidth}{0.8pt}% Footer rule
%\fancyfoot[L]{\scriptsize{Curriculum Vitae - Matheus da Silva Nascimento - \pt{Atualizado em}\fr{Mise à jour}\en{Last Updated in} \abril 2020}}% Left footer
%\fancyfoot[R]{\thepage{} of \pageref{LastPage}}% Right footer


 % \fancyfoot[L]{\scriptsize{Curriculum Vitae - Matheus da Silva Nascimento - \pt{Atualizado em}\fr{Mise à jour}\en{Last Updated in} \marco 2020}}% Left footer
  %\fancyfoot[R]{\thepage{} of \pageref{LastPage}}% Right footer
  
\fancypagestyle{capa}{%
  \fancyhf{}
  \fancyfoot[L]{\scriptsize{Curriculum Vitae -- \nome \sobrenome \hspace{0cm} -- \atualizado}}% Left footer
  \fancyfoot[R]{\thepage{} \en{of}\fr{de}\pt{de} \pageref{LastPage}}% Right footer
}

\fancypagestyle{resto}{%
  \fancyhf{}
  \renewcommand{\headrulewidth}{0pt}
  \fancyhead[C]{\footnotesize{TOC \thepage}}
  \fancyfoot[C]{\footnotesize{TOC \thepage}}
}


% ------------- documento ------------- 
\begin{document}
%\include{paginainicial}
\pagestyle{capa}
%-----       letter       ---------------------------------------------------------
% recipient data
\recipient{Company Recruitment team}{Company, Inc.\\123 somestreet\\some city}
\date{\setembro 18, 2025}
\opening{Dear Sir or Madam,}
\closing{Yours faithfully,}
\enclosure[Attached]{curriculum vit\ae{}} % use an optional argument to use a string other than "Enclosure", or redefine \enclname
\makelettertitle

Lorem ipsum dolor sit amet, consectetur adipiscing elit. Duis ullamcorper neque sit amet lectus facilisis sed luctus nisl iaculis. Vivamus at neque arcu, sed tempor quam.
Curabitur pharetra tincidunt tincidunt. Morbi volutpat feugiat mauris, quis tempor neque vehicula volutpat. Duis tristique justo vel massa fermentum accumsan. Mauris ante
elit, feugiat vestibulum tempor eget, eleifend ac ipsum. Donec scelerisque lobortis ipsum eu vestibulum. Pellentesque vel massa at felis accumsan rhoncus.

Suspendisse commodo, massa eu congue tincidunt, elit mauris pellentesque orci, cursus tempor odio nisl euismod augue. Aliquam adipiscing nibh ut odio sodales et pulvinar
tortor laoreet. Mauris a accumsan ligula. Class aptent taciti sociosqu ad litora torquent per conubia nostra, per inceptos himenaeos. Suspendisse vulputate sem vehicula
ipsum varius nec tempus dui dapibus. Phasellus et est urna, ut auctor erat. Sed tincidunt odio id odio aliquam mattis. Donec sapien nulla, feugiat eget adipiscing sit amet,
lacinia ut dolor. Phasellus tincidunt, leo a fringilla consectetur, felis diam aliquam urna, vitae aliquet lectus orci nec velit. Vivamus dapibus varius blandit.

Duis sit amet magna ante, at sodales diam. Aenean consectetur porta risus et sagittis. Ut interdum, enim varius pellentesque tincidunt, magna libero sodales tortor, ut
fermentum nunc metus a ante. Vivamus odio leo, tincidunt eu luctus ut, sollicitudin sit amet metus. Nunc sed orci lectus. Ut sodales magna sed velit volutpat sit amet pulvinar diam venenatis.

\makeletterclosing
\clearpage

%-----       resume       ---------------------------------------------------------




\makecvtitle
%\textVersion

% ------------- profil ------------- 
%\section{\sectionPerfil}

%\begin{itemize}%
%\item {\fr{2 ans d'expérience en développement WEB et 1 an en programmation de scripts Groovy et SQL}\en{2 years of experience in WEB development and 1 year in programming Groovy and SQL scripts}\pt{2 anos de experiência em desenvolvimento WEB e 1 ano em programação de scripts Groovy e SQL}};
%\item {\fr{Facilité d’apprentissage}\en{Ease to learning}\pt{Facilidade de aprendisagem}};
%\item {\fr{Capacité à travailler en équipe}\en{Ability to work in a team}\pt{Capacidade de trabalho em equipe}};
%\item {\fr{Ouvert à nouveaux défis}\en{Open to new challenges}\pt{Aberto à novos desafios}};
%\item {\textLinguas}
%\end{itemize}


% ------------- competencias ------------- 
%\section{\sectionCompetencia}

%\textLinguagens
%\textFerramentas
%\textBD


% ------------- experiencia ------------- 
\section{\sectionExeperiencia}

\cventry{\cityQC, \countryCN \remoto}{Paradigme3 {\href{https://paradigme3.ca/}{\textcolor{blue}{\faExternalLink}}}}{\pt{Desenvolvedor Full Stack}\fr{Analyste Programmeur }\en{Programmer Analyst}}{\outubro 2022 -- \atual}{\pt{}\fr{Services-conseils en TI}\en{IT Consulting Services}}{}{
  \vspace{-0.4cm}
    \cventry{\junho 2024 -- \atual}{\ \ \ \ \ \ Mandate: Ministère de la Santé et des Services sociaux}{}{}{project: modernization WSTAT}{}{
        \begin{itemize}[leftmargin=1.8cm,  parsep=0cm,  topsep=-0.25cm]
            \item{\en{test}\fr{test}}
        \end{itemize}
        \vspace{0.4cm}
        \hspace{1cm}\ambientenojump{Angular (TypeScript, PrimeNg, Bootstrap), Java 17 (Quarkus, Maven, JUnit, Hibernate), Azure Data Studio, VScode, IntelliJ, Git, Docker, REST, Swagger, Azure Devops, Azure Data Studio, Docker, Kubernetes, Agile, Microsoft Teams. }
    }
    \\
    \cventry{\outubro 2022 -- \junho 2024}{\ \ \ \ \ \ Mandate: Ministère de la Santé et des Services sociaux}{}{}{project: SIPAD}{}{
        \begin{itemize}[leftmargin=1.8cm,  parsep=0cm,  topsep=-0.25cm]
            \item{\en{test}\fr{test}}
        \end{itemize}
        \vspace{0.4cm}
        \hspace{1cm}\ambiente{Java (Maven, Struts 1.3, Tomcat), JMeter, VisualVM, IReports, Eclipse, Git, GitLab, Sourcetree, Oracle SQL Developer, Microsoft Teams, Agile, Jira. }
    }
}

% ------------- experiencia ------------- 
\cventry
{\citySP, \countryBR \remoto}{Stefanini Group {\href{https://stefanini.com/en}{\textcolor{blue}{\faExternalLink}}}}{\pt{Desenvolvedor Full Stack}\fr{Développeur FullStack}\en{FullStack Developer}}{\junho 2021 -- \outubro 2022}{\pt{}\fr{Conseil en stratégie digitale}\en{Global Tech Consultancy}}{}{
        \begin{itemize}[leftmargin=1cm,  parsep=0.1cm,  topsep=-0.25cm]
            \item{\en{test}\fr{test}}
        \end{itemize}
         \vspace{0.4cm}
\hspace{1cm}\ambiente{AngularJS (JavaScript, Bootstrap, Grunt, Karma, JasmineJS), Angular (TypeScript, PrimeNg, PrimeFlex, Karma, JasmineJS), Java (Quarkus, Maven, JUnit), VScode, Eclipse, Git, Gitlab, Swagger, Docker, SonarQube, Highcharts, DBeaver, REST, Agile, VMware Virtual Desktop Infrastructure (VDI), Jenkins, Artifactory, Kubernetes, Microsoft Teams. }
}

% ------------- experiencia ------------- 
\cventry
{\citySP, \countryBR \remoto}{Gauge - Stefanini Group {\href{https://gauge-next-challenge.vercel.app/}{\textcolor{blue}{\faExternalLink}}}}{\pt{Desenvolvedor Full Stack}\fr{Développeur FullStack}\en{FullStack Developer}}{\setembro 2020 -- \junho 2021}{\pt{}\fr{Conseil en stratégie digitale}\en{Digital Consulting}}{}{
        \begin{itemize}[leftmargin=1cm,  parsep=0.1cm,  topsep=-0.25cm]
            \item{\en{test}\fr{test}}
        \end{itemize}
         \vspace{0.4cm}
\hspace{1cm}\ambiente{ReactJS(TypeScript, Bootstrap, Axios, NPM), VScode, Git, Gitflow, Bitbucket, Github, Microsoft SQL Server, Agile, Asana, TaskRow, ServiceNow, JIRA.}
}

% ------------- experiencia ------------- 
\cventry
{\cityBSB, \countryBR}{NExT UnB {\href{http://next.unb.br/}{\textcolor{blue}{\faExternalLink}}}}{{\desenvolvedor} WEB}{\agosto 2018 -- \setembro 2020}{\pt{Núcleo de Pesquisa e Desenvolvimento para Excelência e Transformação do Setor Público (UnB)} \fr{Centre de recherche et développement pour le secteur public (UnB)}\en{Research and Development Center for Excellence and Transformation of the Public Sector}}{}{
        \begin{itemize}[leftmargin=1cm,  parsep=0.1cm,  topsep=-0.25cm]
            \item{\en{test}\fr{test}}
        \end{itemize}
         \vspace{0.4cm}
\ambientenojump{Java (Spring Boot, Spring MVC, Spring Data JPA, Maven, Criteria Query), ReactJS \& React Native (JavaScript, Axios, Bootstrap, NPM), PostgreSQL, REST, Git, Eclipse, VScode, Docker, Postman, Firebase.}\\
\hspace{1cm}\ambiente{Java (Vraptor4, Maven, Wildfly, Hibernate, HSQL, JMeter), ReactJS(JavaScript, Bootstrap, AJAX), REST, MySQL, Git, GitHub, Eclipse, VScode, Postman, Slack, Joomla.}
}

% ------------- experiencia ------------- 
\cventry
{\cityBSB, \countryBR}{Topocart {\href{http://www.topocart.com.br/}{\textcolor{blue}{\faExternalLink}}} }{\programador}{\setembro 2017 -- \agosto 2018}{\pt{Empresa de levantamento de dados, geoprocessamento, topografia}\fr{Entreprise de collecte de données; géotraitement; topographie}\en{Data collection; geoprocessing; topography}.}{}{
        \begin{itemize}[leftmargin=1cm,  parsep=0.1cm,  topsep=-0.25cm]
            \item{\en{test}\fr{test}}
        \end{itemize}
         \vspace{0.4cm}
\hspace{1cm}\ambiente{Groovy (Grails), JavaScript (Jquery, JasmineJS), Selenium, PostgreSQL, PostGIS, Jenkins, Subversion, Git, JasperReports, Qgis, Netbeans, PuTTY, Linux}
}

% ------------- experiencia ------------- 
%\cventry{\cityBSB, \countryBR}{Superior Tribunal de Justiça - STJ {\href{http://international.stj.jus.br/}{\textcolor{blue}{\faExternalLink}}} }{\pt{Estágio}\fr{Stage}\en{Internship}}{\maio 2014 -- \dezembro 2015}{\en{Superior Court of Justice}\fr{Supérieur Court de Justice}\pt{}.}{}{
%    \begin{itemize}[leftmargin=1.8cm,  parsep=0cm,  topsep=-0.25cm]
%    \item {\pt{Acompanhar o suporte a usuários e clientes na utilização de software}\fr{Accompagner le support aux utilisateurs et aux clients dans l'utilisation du logiciel}\en{Support the users and customers in the use of the software}};
%    \item {\pt{Acompanhamento das instalações de redes, micros e comunicação de dados}\fr{Accompagnement d'installation de réseaux, d'ordinateurs et de la communication de données}\en{Support for the installation of networks, computers and data communication}};
%    \item {\pt{Realizar manutenção de equipamentos conectados em rede}\fr{Réaliser la maintenance d'équipements connectés à internet}\en{Carry out maintain equipment connected to the internet}};
%    \end{itemize}}


% ------------- Formação escolar ------------- 
\section{\sectionFormacao}
\textUniv
%\textEscola

% ------------- Línguas ------------- 
%\section{Languages}
%\cvitemwithcomment{Portugais}{Natif}{}
%\cvitemwithcomment{Anglais}{ Intermediaire}{}
%\cvitemwithcomment{Français}{Debutant}{}

\section{\sectionComplementar}
\textComplementTres
\textComplementDois
\textComplementUm
%\cventry{2017}{\fr{Certificate de informatique, Caelum (Bréseil)}\en{Certificate, Caelum (Brazil)}}{{\desenvolvedor} WEB}{}{}{}



% ----------------------------------------------
%\hfill \break

\nocite{*}
\bibliographystyle{plain}
\bibliography{publications}
\end{document}


















%Concevoir
%Développer
%Effectuer
%Réaliser
%Former
%Programmer
%Générer
%Appoyer




% https://github.com/xdanaux/moderncv/blob/master/manual/moderncv_userguide.v2.pdf
%https://github.com/xdanaux/moderncv/blob/master/examples/template.tex
% file:///E:/Downloads/fontawesome.pdf
% ou http://linorg.usp.br/CTAN/fonts/fontawesome/doc/fontawesome.pdf

%\section{Autres Informations}
%\begin{cvcolumns}
%  \cvcolumn{Category 1}{\begin{itemize}\item Person 1\item Person 2\item Person 3\end{itemize}}
%  \cvcolumn{Category 2}{Amongst others:\begin{itemize}\item Person 1, and\item Person 2\end{itemize}(more upon request)}
%  \cvcolumn[0.5]{All the rest \& some more}{\textit{That} person, and \textbf{those} also (all available upon request).}
%\end{cvcolumns}

%exemplo de seção
%\section{Profil}
%\cvitem{category 1}{XXX, YYY, ZZZ}{category 4}{XXX, YYY, ZZZ}
%\cvdoubleitem{category 2}{XXX, YYY, ZZZ}{category 5}{XXX, YYY, ZZZ}
%\cvitemwithcomment{Portugais}{Natif}{}
%\cventry{período}{empresa}{cargo}{cidade,país}{}
%{resumo empresa.\newline{}\hyperlink{link/}{\textcolor{blue}{same link}}\newline{}%
%\begin{itemize}
%\item Programmer 2, with sub-achievements:
% \begin{itemize}%
% \item Effectuer consultas em postgresql/postgis;
% \end{itemize}
%\end{itemize}
%environnements: etc.
%\newline{}%}





%opções de formatação da página
%\documentclass[12pt,a4paper,sans]{moderncv}        % possible options include font size ('10pt', '11pt' and '12pt'), paper size ('a4paper', 'letterpaper', 'a5paper', 'legalpaper', 'executivepaper' and 'landscape') and font family ('sans' and 'roman')

% modern themes
%\moderncvstyle{banking} % style options are 'casual' (default), 'classic', 'oldstyle' 'fancy' and 'banking'
%\moderncvcolor{red}                                % color options 'blue' (default), 'orange', 'green', 'red', 'purple', 'grey' and 'black'
%\renewcommand{\familydefault}{\sfdefault}         % to set the default font; use '\sfdefault' for the default sans serif font, '\rmdefault' for the default roman one, or any tex font name
%\nopagenumbers{}

% character encoding
%\usepackage[utf8]{inputenc}                       % if you are not using xelatex ou lualatex, replace by the encoding you are using
%\usepackage{CJKutf8}                              % if you need to use CJK to typeset your resume in Chinese, Japanese or Korean

% adjust the page margins
%\usepackage[scale=0.75]{geometry}
%\setlength{\hintscolumnwidth}{3cm}                % if you want to change the width of the column with the dates
%\setlength{\makecvtitlenamewidth}{10cm}           % for the 'classic' style, if you want to force the width allocated to your name and avoid line breaks. be careful though, the length is normally calculated to avoid any overlap with your personal info; use this at your own typographical risks...





% Publications from a BibTeX file without multibib
%  for numerical labels: \renewcommand{\bibliographyitemlabel}{\@biblabel{\arabic{enumiv}}}% CONSIDER MERGING WITH PREAMBLE PART
%  to redefine the heading string ("Publications"): \renewcommand{\refname}{Articles}
%\nocite{*}
%\bibliographystyle{plain}
%\bibliography{publications}                        % 'publications' is the name of a BibTeX file
% Publications from a BibTeX file using the multibib package
%\section{Publications}
%\nocitebook{book1,book2}
%\bibliographystylebook{plain}
%\bibliographybook{publications}                   % 'publications' is the name of a BibTeX file
%\nocitemisc{misc1,misc2,misc3}
%\bibliographystylemisc{plain}
%\bibliographymisc{publications}                   % 'publications' is the name of a BibTeX file




%\vspace{5mm}
%Inserts a vertical spaces whose length is 5mm. Other LATEX units can be used with this command.

%\vfill
%Inserts a blank space that will stretch accordingly to fill the vertical space available. That's why the line "Text at the bottom of the page." is moved to the bottom, and the rest of the space is filled in.There are other three commands commonly used to insert vertical blank spaces

%\smallskip
%Adds a 3pt space plus or minus 1pt depending on other factors (document type, available space, etc)
%\medskip
%Adds a 6pt space plus or minus 2pt depending on other factors (document type, available space, etc)
%\bigskip
%Adds a 12pt space plus or minus 4pt depending on other factors (document type, available space, etc)